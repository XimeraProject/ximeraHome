\documentclass{ximera}

\title{Authors}

\begin{document}
\begin{abstract}
\end{abstract}
\maketitle

An author of Ximera material just writes \LaTeX\ code. 
Each webpage is generated from a \LaTeX\ document with documentclass \verb|ximera|, which is an extension of article that defines some extra commands and environments that e.g. generated interactive anwser boxes.

A course is just a collection of webpages, and is created as a \LaTeX\ document of documentclass \verb|xourse|.

More info is found on \href{https://github.com/XimeraProject}{github}, 
especially in the \href{https://github.com/XimeraProject/ximeraFirstSteps}{ximeraFirstSteps} repository that lets you start building and publishing a course right from within your browser.
The only thing you need to get started is a free github account.

An integration with Overleaf is possible, and will be documented soon.
Of course you can also work locally on your own PC if you prefer.

There is a \href{https://github.com/XimeraProject/ximeraManuals/releases/download/v1.5.0/ximeraUserManual.pdf}{PDF manual}, 
written in Ximera and thus also available 
\href{https://set-p-dsb-zomercursus-latest.cloud-ext.icts.kuleuven.be/xmanual/ximeraUserManual/introductionAndSetup/aboutXimera}{online}, and 
on the (legacy) \href{https://ximera.osu.edu/xmanual/ximeraUserManual/introductionAndSetup/aboutXimera}{OSU Ximera Server}.

\end{document}

